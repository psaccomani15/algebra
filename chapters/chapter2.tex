\documentclass[../main]{subfiles}

\begin{document}
\newcommand{\stmat}[3]{\begin{pmatrix} 
    1 & 2 & 3 \\
    #1 & #2 & #3 \end{pmatrix}}
\subsection{Definições e Propriedades}
Daremos uma definição inicial em termos de objetos já trabalhados:
\begin{definition}
    Um grupo é um grupóide com um único objeto.
\end{definition}

Isto é, um grupóide com um único objeto é uma categoria na qual, denominando por
\(\varphi\) esse único objeto, então \(\text{Aut}_\cat(\varphi)\) carrega toda
informação sobre ela. Denominamos assim um grupo \(G = Aut_\cat(\varphi)\).
Assim, diretamente, vemos que pela definição de categoria temos a existência de
uma identidade, sobre uma operação associativa (i.e morfismos), além disso como
todo morfismo é iso, temos também a existência da inversa.

Dada essa prévia discussão, em termos usuais, podemos pensar em um grupo como um
conjunto munido de uma operação obedecendo certos axiomas, isto é:
\begin{definition}[Grupo]
    Seja \(G\) um conjunto não vazio, munido de uma operação binária \(\cdot: G
    \times G \to G\) denotado pela tupla \((G, \cdot)\), então \(G\) é um grupo
    caso:
    \begin{itemize}
        \item A operação \(\cdot\) é associativa, ou seja:
        \[
            \forall g, h, k \in G, (g \cdot h) \cdot k = g \cdot (h \cdot k)
        \]
        \item Existência da identidade \(e_G\) para a operação:
        \[\exists e_G \in G \forall g \in G, g \cdot e_g = e_g \cdot g = g\]
        \item Existência da inversa:
        \[\forall g \in G, \exists h \in G, hg = gh = e_G\]
    \end{itemize}
\end{definition} 

\begin{exercise}
    Mostre que as duas definições são de fato equivalentes, isto é mostre que
    todo grupo pode ser escrito como o grupo de automorfismos de algum objeto em
    uma categoria.
\end{exercise}
\begin{answer}
    Seja \(G\) um grupo. Construímos a categoria \(\cat\) da seguinte forma:
    Tome \(\Obj(\cat) = e_G\). e considere 
\end{answer}
Com essas diferentes visões sobre um grupo dadas, podemos começar a tentar obter
propriedades interessantes sobre essas estruturas, assim:
\begin{prop}
    A identidade é única. Isto é, se \(h \in G\) é uma identidade, então \(h =
    e_G\) assim concluímos que todo grupo possuí ao menos um elemento especial
    bem definido.
\end{prop}
\begin{proof}
    Por inspeção:
    \[h = e_G h = e_g\]
\end{proof}

\begin{prop}
    Elementos inversos também são únicos.
\end{prop}
\begin{proof}
    Tome \(g \in G\) e \(h_1, h_2 \in G\) inversos de \(g\), novamente, por
    inspeção:
    \[h_1 = e_G h_1 = h_2(h_1 g) = h_2\]
\end{proof}
\begin{prop}
    Seja \((G, \cdot)\) um grupo. Então \(\forall a, g, h \in G\) \(ga = ha
    \implies g = h, ag = ah \implies g = h\). 
\end{prop}
\begin{proof}
    Novamente, por inspeção:
    \[ga = ha \implies (ga)a^{-1} = (ha)a^{-1} \implies g = h\]. Da mesma forma,
    concluímos o mesmo para o outro fato.
\end{proof}

\subsubsection*{Grupos comutativos}

Até agora em nenhum momento exigimos que elementos de um grupo comutem sobre a
operação, nem sequer definimos o significado de "comutar", entretanto grupos
comutativos são de extrema importância matemática. Assim:
\begin{definition}
    Dizemos que um grupo é comutativo (ou abeliano) se:
    \[\forall g, h \in G, gh = hg\]    
\end{definition}

Quando um grupo é abeliano, mudamos nossa terminologia. Usualmente se trata a
operação conforme uma adição, denotando a identidade como \(0_G\), \(\cdot\)
como \(+\) e a inversa de \(a\) por \(-a\).

\subsection*{Ordem}
\begin{definition}
    Um elemento \(g \in G\) possuí ordem finita se \(g^n = e\) para algum n
    inteiro positivo. Nesse caso, a ordem \(|g|\) é o menor inteiro \(n\) tal
    qual isso é observado.
\end{definition}

Dessa definição abstraímos diretamente:
\begin{lemma}
    Se \(g^n = e\) para algum n inteiro, então \(|g|\) divide \(n\). 
\end{lemma}
\begin{lemma}
  Temos que \(g^n = e\), para \(m = |g|\), \(g^m = e\) tal qual \((g^m)^p = e,
  \forall p \in \mathbb Z^+\) com necessariamente \((g^m)^q = (g^n)\) para algum
  \(q\), já que caso contrário, podemos verificar \(m\) não seria o menor
  inteiro assim diretamente \(n = m * q\), portanto \(m \text{divide } n \iff
  |g| \text{divide } n\). 
\end{lemma}

\begin{definition}
    Se G enquanto conjunto é finito, então a ordem \(|G|\) de \(G\) é sua
    cardinalidade, caso \(G\) é infinito, então \(|G| = \infty\)
\end{definition}

Dessa definição podemos concluir que \(|g| \leq |G|, \forall g \in G\), já que
no caso finito (infinito é vacuosamente verdadeiro) para todo \(g \in G\), se
\(n = |G|\) então \((g^0, ..., g^n)\) não podem ser todos distintos como
consequência direta dessa finitude, então, \(\exists i, j: 0 \leq i < j \leq
|G|\), assim \(g^i = g^j \implies g^{j - i} = e\). Dessa forma \(|g| \leq (j -
i) \leq |G|\). 

Desse fato, concluímos também que se \(g\) possuí ordem finita, então \(g^m,
\forall m \geq 0\) também e:
\[ |g^m| = \frac{\text{lcm}(m, |g|)}{m} = \frac{|g|}{\gcd(m, |g|)}\]

Tal conclusão segue do fato que inicalmente \(\text{lcm}(a, b) =
ab/\gcd(a, b), \forall a, b\). Assim é necessário mostrar que \(|g^m| =
\frac{\text{m, |g|}}{m}\). Assim, tome \(d\) tal qual \(g^{md} = e\), ou seja d
tal qual \(md = \alpha |g|\), dessa forma por essa definição \(m|g^m|\) é o
menor multiplo de m e de \(|g|\) assim, \(m|g^m| = \text{lcm}(m, |g|)\),
concluindo o nosso objetivo.

Em geral, para elementos comutativos, \(gh = hg \implies |gh| \text{ divide
lcm}(|g|, |h|)\). O que pode ser demonstrado tomando \(|g| = m, |h| = n\) tal
qual se \(N\) é múltiplo comum de \(m, n\) então \(g^N = h^N = e\)

\subsection{Exemplos de Grupos}
\begin{definition}[Grupos Simétricos]
    Seja \(A\) um conjunto. O grupo simétrico, ou grupo de permutações  de \(A\)
    é o grupo \(\text{Aut}_{\mathfrak{Set}}(A)\). O grupo de permutações do
    conjunto \(\{1, ..., n\}\) é denotado por \(S_n\)
\end{definition}

Essa ideia de permutação pode ser observada mais claramente, ao observar o
comportamento desse conjunto de automorfismos. Inicialmente fixamos um conjunto
\(A\), temos então que como \(\cat = \mathfrak{Set}\), \(\text{Aut}(A)\) são
simplesmente as bijeções de \(A\) para \(A\), ou seja, suas permutações, cuja
simetria está ligado a uma apenas "reordenação" do conjunto preservando sua
propriedades. 

Nessa definição, algo que deixou-se de lado a forma em que pensamos no produto
\(\cdot\). Na visão categorial, a definição direta advém da composição de
morfismos, isto é defimos:
\[f \cdot g := \forall p \in A, g \circ f(p) = g(f(p)).\]

Um jeito de pensar nos elementos dos grupos \(S_n\), é utilizando a notação por
matrizes \(2 \times n\) com \(f \in Aut_{\mathfrak{Set}}(A)\):
\[a_{ij} = \begin{cases} i = 1, a_{ij} = j i = 2, a_{ij} = f(j) \end{cases}\]

De forma auxiliar a visualização. Podemos pensar também no produto como
\(\forall p \in A, fp = (p)fp\), nessa notação também sendo facilitado. 

Dando atenção em específico a \(S_3\) vemos que enquanto conjunto ele é dado
por:
\begin{align*}
    \left \{ \stmat{1}{2}{3}, \stmat{2}{1}{3}, \stmat 321 \stmat 132 \stmat 312 \stmat 231 \right \}
\end{align*}
\begin{exercise}
    Mostre que para \(n \geq 3, S_n\) não é comutativo.
\end{exercise}

\begin{definition}[Grupos diedrais]
    Grupos surgem naturalmente no contexto de simetrias, imediatamente a partir
    da definição. Isto é, formalmente, pensamos em simetrias como transformações
    que preservam certas propriedades de um objeto matemático, isto é,
    automorfismos de dado objeto de uma categoria. 
    
    Assim, no contexto de figuras geométricas (planas) surge o conceito de
    grupos diedrais, tal qual definimos grupos diedrais como o conjunto de todas
    transformações espaciais que ocasionam em simetriais (movimentos rigidos). 

    Assim, para um polígono regular de \(n\) lados centrado na origem, seu grupo
    simétrico é composto por n rotações de \(\frac{2\pi}{n}\) ao longo da origem
    e \(n\) reflexões dadas por seus eixos de simetria. Dessa forma denominamos
    o grupo de simetrias associadas a um polígono regular de \(n\) lados como
    \(D_{2n}\).
\end{definition}

É interessante ver que, dessa definição, concluímos diretamente que há uma
relação entre grupos diedrais e simétricos, isto é, mais especificamente, existe
uma função \(f: D_{2n} \to S_n\), injetiva (confira!),  basta ver que \(D_{2n}\)
pode ser caracterizado enumerando cada vertice e suas transformações simétricas,
que devem ser um subconjunto das permutações possíveis. Mais adiante trataremos
tais casos com mais rigor e generalizações.

\begin{definition}[Grupos Cíclicos]
    Iniciamos esse caso, tomando a seguinte relação de equivalência:
    \[(\forall a, b \in \mathbb Z): a \equiv b \mod n \iff n | (b - a)\]
    Denominada congruência módulo \(n\), denotada por \(\modset{n}\) É evidente
    que \(\modset{n}\) possuí \(n\) elementos (restos possíveis para essa
    divisão):
    \[[0]_n, [1]_n, ..., [n - 1]_n\]

    Definimos assim, a relação binária nesse conjunto \(+: \modset n \times
    \modset n \to \modset n, \forall a, b \in \mathbb Z, [a] + [b] := [a + b]\),
    de modo a obter um grupo a partir dessa estruturas (verifique!).
    
    É evidente ainda, como essa operação herda propriedades de \(\mathbb Z\),
    esse grupo é abeliano.

    Grupos cíclicos terão maior enfase posteriormente, mas sua importância deve
    ser claramente notada, por sua estrutura cíclica e modular. 
\end{definition}

\begin{prop}
    Todo grupo cíclico definido da forma dada anteriormente é gerado por \([1]_n
    \in \modset n\).
\end{prop}
\begin{proof}
    É fácil ver que \(|[1]_n| = n\), assim como tal grupo possuí \(n\)
    elementos, e como todo \(m [1]_n, 0 \leq m < n, m \in \mathbb Z\) são
    distintos dois a dois, \([1]_n\) gera esse grupo.
\end{proof}
\begin{prop}
    Da proposição anterior, vemos que a ordem de qualquer \([m]_n \in \modset
    n\) é \(1, n | m\) e caso contrário:
    \[|[m]_n| = \frac{n}{\gcd(m, n)}\]
\end{prop}

Essa última proposição nos revela uma propriedade ainda mais importante desses
grupos:
\begin{corol}
    A classe \([m]_n\) gera \(\modset n\) se só se \(\gcd(m, n) = 1\)   
\end{corol}

Esse fato se torna ainda mais relevante, quando consideramos o caso em que \(n =
p\) tal qual \(p\) é primo, já que toda classe de equivalência não nula geraria
esse grupo.

Temos ainda que, tomando \((\modset n)^* := \{[m]_n \in \modset n |
\gcd(m, n) = 1\}\) munido da operação de multiplicação \([a]_n, [b]_n \in
(\modset n)^*, [a]_n \cdot [b]_n = [a \cdot b]_n\) é um grupo. 

Esse fato pode ser observado já que \(\gcd(m_1, n) = \gcd(m_2, n) = 1 \implies \gcd(m_1m_2, n) = 1\) (operação bem definida), cuja associatividade e identidade são resultados da multiplicação nos inteiros. Agora para observar a existência da inversa, vemos que como todo elemento desse grupo gera o grupo aditivo \(\modset n\), então:
\(\exists a \in \mathbb Z, a \cdot [m]_n = [1]_n\)
\end{document}