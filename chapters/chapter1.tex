\documentclass[../main]{subfiles}

\begin{document}
\subsection{Princípios de Teoria de Conjuntos}
Antes de iniciar a discussão, é interessante tratar algumas definições básicas e
iniciais, assim temos:

\begin{definition}[Relação de Equivalência]
    Uma relação de equivalência sobre um conjunto \(S\) é uma relação \(~\) que
    obedece as seguintes propriedades:

    \begin{itemize}
        \item Reflexividade: \((\forall a \in S) a \sim a\).
        \item Simetria: \((\forall a \in S) (\forall b \in S) a \sim b \implies
        b \sim a\)
        \item Transitividade: \((\forall a, b, c \in S)(a \sim b \text{ and } b
        \sim c) \implies a \sim c\)
\end{itemize}
\end{definition}
Dessa definição surge a possibilidade de se obter uma partição de \(S\). Assim,
se \(\forall a \in S \) a classe de equivalência de \(a\) é definida como:

\[[a]_\sim := \{b \in S | b \sim a\} \]

\begin{exercise} 
    Mostre que as classes de equivalência formam uma partição \(\mathcal P_\sim
    \) de \(S\)
\end{exercise}
\begin{answer}
    Para demonstrar esse item queremos mostrar que cada conjunto da classe
    \(\mathcal P_\sim \) são disjuntos dois a dois, e \(\bigcup \mathcal P =
    S\).
    
    Dessa forma, tomando \(A, B \in \mathcal P_\sim \), tal qual \(A \cap B \neq
    \emptyset \), temos \(\exists x \in A \cap B\) tal qual, \(\forall y \in A,
    x ~ y \implies y \in B \implies A \subset B\) e \(\forall \overline y \in B,
    x ~ \overline y \implies \overline y \in A \implies B \subset A \therefore A
    = B\). Assim, de fato, todo conjunto nessa classe deve ser disjunto entre
    sí. 
    
    É evidente que \(\bigcup \mathcal P = S\), diretamente da definição de
    classes de equivalência. 
\end{answer}
\begin{exercise}
    Mostre que cada partição \(\mathcal P\) possuí uma relação de equivalência
    correspondente, concluindo assim, que as noções de partição e relação de
    equivalência são as mesmas.
\end{exercise}

A partir dessa ideia de que a classe de conjuntos formada pelas classes de
equivalência de \(S\) formam uma partição, podemos partir para a ideia de
operação de quociente em um conjunto:

\begin{definition}[Quociente]
    Seja \(S\) um conjunto, então seu quociente em respeito a relação de
    equivalência \(\sim \) é o conjunto:
    \[S/\sim := \mathcal P_\sim \]
\end{definition}
\begin{example}
    Seja \(S = \mathbb Z\) e \(\sim \) definida por:
    \[a \sim b \iff a - b \text{ é par.}\]

    Então \(\mathbb Z/\sim \) consiste das duas classes de equivalência,
    nominalmente:
    \[\mathbb Z/\sim = \{[0]_\sim, [1]_\sim\}.\]

    Que em linhas gerais, "separa" os inteiros em pares e ímpares (motivação
    para aritmética modular).
\end{example}

    Uma forma de se pensar sobre a operação de quociente, é que neste novo
    conjunto, a relação de equivalência se torna uma igualdade nesse conjunto.

\subsection{Funções entre conjuntos}

Além dos exemplos já conhecidos e tratados, temos também algumas "novas" noções,
tais quais:

\begin{definition}[Monomorfismo]
    Uma função \(f : A \to B\) é um monomorfismo (ou \textit{monic}) se para
    todos conjuntos \(Z\) e todas funções \(\alpha', \alpha'': Z \to A\), \(f
    \circ \alpha' = f \circ \alpha'' \implies \alpha = \alpha''\) 
\end{definition}

\begin{prop}
    Uma função é injetiva se só se é um monomorfismo.
\end{prop}
\begin{proof}
    (\(\implies\)) Sabemos que uma função \(f: A \to B\), é injetiva então
    possuí inversa à esquerda, i.e \(\exists g, g \circ f = id_A\). Assim, seja
    \(\alpha'', \alpha'', Z \to A, f \circ \alpha' = f \circ \alpha''\). Da
    hipótese e do resultado enunciado, tomando a composição dessa construção por
    g, e a associatividade: 
    \[(g \circ f) \circ \alpha' = (g \circ f) \circ \alpha''\]

    Assim, \(id_A \circ \alpha' = id_A \circ \alpha'' \implies \alpha' =
    \alpha''\). Assim, concluímos f é um monomorfismo.

    Para a (\(\impliedby \)) partimos de uma \(f\) monomorfismo. Pela definição
    podemos tomar \(Z = \{p\}\) para um dado \(p\), e \(\alpha', \alpha'' Z \to
    A\). 
    
    Para toda temos que se \(\alpha'' = \alpha''\) então para qualquer escolha
    de \(\alpha'(p), \alpha''(p)\), vale que \(f(\alpha') = f(\alpha'') \implies
    \alpha' = \alpha''\). Pela contrapositiva, i.e \(\alpha' \neq \alpha''
    \implies f(\alpha') \neq f(\alpha'')\). Assim, \(f\) deve ser injetiva. 
\end{proof}

\begin{exercise}
    Formule a noção de epimorfismo de maneira análoga a de monomorfismo e
    formule e demonstre um resultado análogo ao anterior.
\end{exercise}

\begin{example}
    Sejam A e B dois conjuntos. Então temos as projeções naturais, \(\pi_A,
    \pi_B\): 
    \begin{center}
    \begin{tikzcd}[column sep = small]
          & A \times B \arrow[two heads]{dr}{\pi_B} \arrow[two
          heads]{dl}[swap]{\pi_A} &  \\
        A & & B    
    \end{tikzcd}
\end{center}
    Definidas por \(\pi_A = ((a, b)) := a, \pi_B((a, b)) := b\), \(\forall (a,
    b) \in A \times B\). Tais quais por definição são sobrejetores.
\end{example}
\begin{example}
    Similarmente, existem funções injetoras naturais de A para B para a união
    disjunta (denotada aqui por \(\amalg\)):
    \begin{center}
        \begin{tikzcd}[column sep = small]
            A \arrow[hook]{dr} & & B \arrow[hook']{dl} \\
            & A \amalg B &  
        \end{tikzcd}
    \end{center}

    Obtida ao associar cada elemento de \(a, b \in A, B\) a seus respectivos
    elementos de \(A + B\).
\end{example}


\begin{example}
    Seja \(\sim \) uma relação de equivalência sobre um conjunto \(A\), então
    existe uma projeção canônica, sobrejetora:
    \begin{center}
    \begin{tikzcd}
        A \arrow[two heads]{r} & A/\sim 
    \end{tikzcd}
    \end{center}

    Obtido ao associar todo \(a \in A\) a sua classe de equivalência
    \([a]_\sim\).
\end{example}

Até agora foi dada uma clara enfâse aos mapas injetores e sobrejetores, isso
ocorre pois elas conferem os "blocos" construtores necessários para qualquer
função. Para observar este fato, basta observar que cada função \(f: A \to B\)
determina uma relação de equivalência em A tal qual \(\forall \alpha', \alpha''
\in A\),' 
\[\alpha' \sim \alpha'' \iff f(\alpha') = f(\alpha'').\]
\begin{prop}
    Seja \(f: A \to B\) uma função, e \(\sim\) uma relação de equivalência.
    Então f pode ser decomposta como:
    \begin{tikzcd}
        A \arrow[two heads]{r} \arrow[bend left]{rrr}{f} & (A/\sim)
        \arrow{r}{\sim}[swap]{\tilde f} & \text{Im} f \arrow[hook]{r}  & B
    \end{tikzcd}

Onde a primeira função é a projeção canônica e a terceira a operação de inclusão
e \(\tilde f\), bijeção é definida como \(\tilde f([a]_\sim) := f|(a)\)
\end{prop}

\subsection{Categorias}

No contexto atual, vemos a necessidade de dar importância a como dois diferentes
objetos podem se relacionar, como já dada a motivação por conjuntos. Assim, urge
a ideia de pensar em categorias.

\begin{definition}[Categorias]
De forma direta, uma categoria \(\cat\) consiste em uma coleção (classe) de
objetos \(\Obj \cat\) e morfismos entre eles, isto é para todo \(A, B \text{ de
} \cat\), temos um conjunto \(\Hom_{\cat}(A, B)\) de morfismos com as seguintes
propriedades:

\begin{itemize}
    \item Existência da identidade: Para todo objeto \(A\) de \(\cat\) existe ao
    menos um morfismo \(1_A \in \Hom_{\cat}(A, A)\).
    \item Composição de morfismos: Sejam \(f \in \Hom_{\cat} (A, B), g \in
    \Hom_\cat(B, C)\) então \(gf \in \Hom_\cat(A, C)\). Ou seja para toda tripla
    de objetos \(A, B, C, \exists\) h tal qual:
    \[h: \Hom_\cat (A, B) \times \Hom_\cat (B, C) \to \Hom_\cat (A, C)\]
    \item Composição é associativa, isto é, se \(f \in \Hom_\cat(A, B), g \in
    \Hom_\cat(B, C), h \in Hom_\cat (C, D)\), então:
    \[(hg)f = h(gf).\]
    \item As identidade são identidades em relação a composição, isto é para
    todo \(f \in \Hom_\cat (A, B)\)
    \[f1_A = f = 1_B f\]
    \item Se \(A \neq B, B \neq D\) então \(\Hom_\cat(A, B), \Hom_\cat(C, D)\)
    são disjuntos;
\end{itemize}
\end{definition}

Denominamos como \textit{endomorfismo} um morfismo de uma objeto \(A\) da
categoria para sí mesmo, assim \(\Hom_\cat (A, A)\) é denotado por
\(\text{End}_\cat(A)\). 

\begin{example}
    Denotaremos, aqui como \(\mathfrak{Set}\) a categoria de todos conjuntos,
    assim: 
    \begin{itemize}
        \item \(\Obj(\mathfrak{Set} = )\) classe de todos conjuntos.
        \item Para \(A, B \in \Obj(\mathfrak{Set})\), então
        \(\Hom_{\mathfrak{Set}(A, B)} = B^A\)
    \end{itemize}
\end{example}
\begin{example}
    Considere a categoria \(\cat\) que abriga gerada a partir de \(\mathbb Z\)
    munida da relação de equivalência \(\leq\), como o dado seguinte diagrama
    \textit{comutativo}:

    \begin{center}    
    \begin{tikzcd}
        2 \arrow{r} \arrow{dr} & 3 \arrow{r} \arrow{d}{1_3} & 5 \arrow{r} & 7 \\ 
        & 3 \arrow{urr} \arrow{r} & 4 \arrow{ur}
    \end{tikzcd}
    \end{center}
\end{example}
\begin{example}[Slice Category]
    Seja \(\cat\) uma categoria, e \(A\) um objeto de \(C\). Definimos a
    categoria \(\cat_A\) com seus objetos sendo certos morfismos de \(\cat\) e
    seus morfismos correspondentes, diagramas de \(\cat\), isto é:
    \begin{itemize}
        \item \(\Obj (\cat_A) =\) todos morfismos de qualquer outro objeto de
        \(\cat\) para \(A\), i.e, qualquer objeto de \(\cat_A\) é um morfismo
        \(f \in \Hom_\cat(Z, A)\) para algum \(Z\) de \(\cat\), em um diagrama:
        \begin{center}
            \begin{tikzcd}
                Z \arrow{d}{f} \\
                A
            \end{tikzcd}
        \end{center}
        \item Os morfismos dessa categoria podem ser representador pelo seguinte
        diagrama, onde \(f_1, f_2\) são objetos dela:
        \begin{center}
            \begin{tikzcd}
                Z_1 \arrow{rr}{\sigma} \arrow{dr}[swap]{f_1} & & Z_2
                \arrow{dl}{f_2} \\
                & A & 
            \end{tikzcd}
        \end{center}

        Tal qual \(\sigma\) representa o morfismo \(f_1 \to f_2\), com
        \(\cat_A\) denominada a categoria ambiente de \(\cat\).
    \end{itemize}
\end{example}

\begin{exercise}
    Verifique que a categoria descrita anteriormente obedece os respectivos
    axiomas.
\end{exercise}
\begin{example}
    Seja \(\cat\) uma dada categoria, e \(A, B\) dois objetos da mesma. Podemos
    definir uma categoria \(C_{A, B}\) da mesma maneira que definimos no último
    exemplo, isto é:
    \begin{itemize}
        \item \(\Obj(\cat_{A, B})\) são os diagramas:
        \begin{center}
            \begin{tikzcd}
                & A \\ Z \arrow{ur}{f} \arrow{dr}[swap]{g} \\ & B
            \end{tikzcd}
        \end{center}

        Isto é, \(\Obj \cat_{A, B}\) são simplesmente as tuplas formadas tomando
        para todo objeto Z de \(\cat\), \(f \in \Hom(Z, A), g \in \Hom(Z, B),
        (f, g)\).
        \item Seus morfismos podem ser representados pelos diagramas:
        \begin{center}
            \begin{tikzcd}
                
            & A & & & A \\
            Z_1 \arrow{ur}{f_1} \arrow{dr}[swap]{g_1} & \arrow{rr} & & Z_2
            \arrow{ur}{f_2} \arrow{dr}[swap]{g_2} \\
            & B & & & B 
        \end{tikzcd}
        \end{center}
        Que equivalem ao seguinte diagrama comutativo:
        \begin{center}
            \begin{tikzcd} 
                & & A \\
                Z_1 \arrow{r}{\sigma} \arrow[bend left]{urr}{f_1} \arrow[bend
                right]{drr}[swap]{g_1} & Z_2 \arrow{ur}{f_2}
                \arrow{dr}[swap]{g_2} \\
                & & B
            \end{tikzcd}
        \end{center}
        Formalmente podemos pensar nesses diagramas como a definição, tomando
        dois objetos \(C, D\)de \(\cat_{A, B}\), com \(C, D\) sendo morfismos de
        dado \(Z_1, Z_2\) para \(A, B\) de \(\cat\), com \(C = (f_1, g_1), D =
        (f_2, g_2)\) tal qual \(Hom(C, D) = \sigma: Z_1 \to Z_2\) tal qual \(f_1
        = f_2\sigma\) e \(g_1 = g_2 \sigma\) 
    \end{itemize}
\end{example}
    \begin{example}
        Trataremos aqui de \(\cat^{A, B}\). Seja \(\cat\) e fixe dois morfismos
        \(\alpha: A \to C, \beta: B  \to C\) em \(\cat\). Então considere a
        categoria determinada por:
        \begin{itemize}
            \item \(\Obj (\cat_{\alpha, \beta})\) os diagramas comutativos:
            \begin{center}
                \begin{tikzcd}
                    & A \arrow{dr}{\alpha} & \\
                    Z \arrow{ur}{f} \arrow{dr}[swap]{g} & & C \\
                    & B \arrow{ur}[swap]{\beta}
                \end{tikzcd}
            \end{center}
            \item Os morfismos determinados pelo seguinte diagrama:
            \begin{center}
                \begin{tikzcd}
                    & & A \arrow{dr}{\alpha} \\
                    Z_1 \arrow[bend left]{urr}{f_1} \arrow{r}{\sigma}
                    \arrow[bend right]{drr}[swap]{g_1} & Z_2 \arrow{ur}{f_2}
                    \arrow{dr}[swap]{g_2} & & C \\
                    & & B \arrow{ur}[swap]{\beta}
                \end{tikzcd}
            \end{center}
        \end{itemize}
    \end{example}
\begin{exercise}
    Seja \(\cat\) uma categoria. Considere a estrutura \(\cat^{op}(A, B)\) com:
    \begin{itemize}
        \item \(\Obj(\cat^{op}) := \Obj(\cat)\);
        \item Para \(A, B\) objetos de \(\cat^{op}\), \(\Hom_{\cat^{op}}(A, B)
        := \Hom_\cat(B, A)\).
    \end{itemize}


    Mostre como formalizar essa estrutura em uma categoria, isto é defina a
    composição de morfismos e verifique os axiomas necessários.
\end{exercise}

\begin{exercise}
    Seja \(A\) um conjunto finito, então quão grande é
    \(End_{\mathfrak{Set}}(A)\)? 
\end{exercise}
\begin{answer}
    O conjunto de morfismos na categoria \(\mathfrak{Set}\) consiste em todas
    funções \(f: S \to \tilde S\) com \(S, \tilde S\). Assim, podemos pensar em
    no conjunto de endomorfismos de um dado \(A\) de cardinalidade finita, como
    permutações para todos subconjuntos deles. Isto é, denotando M como esse
    valor:
    \begin{align*}
        M = (|A|)^{|A|}
    \end{align*} 
    Tratando esse problema com um de contagem, onde basicamente cada elemento do
    conjunto pode ser e para qualquer outro.
\end{answer}
\begin{exercise}
    Seja \(V\) uma categoria com \(\Obj(V) = \mathbb N\), com \(\Hom(n, m)\) o
    conjunto das matrizes \(m \times n\) com valores reais. 
\end{exercise}

\subsection{Morfismos}

\begin{definition}[Isomorfismo]
    Um morfismo \(f \in \Hom_\cat (A,B)\) é denominado isomorfismo se possuí uma
    inversa por ambos lados, isto é \(\exists g \in \Hom_c(B, A)\) tal qual:
    \[gf = 1_A, fg = 1_B\]
\end{definition}
\begin{prop}
    A inversa de um isomorfismo é única.
\end{prop}
\begin{proof}
    Queremos mostrar que se \(g_1, g_2: B \to A\) atuam como inversas de um
    isomorfismo \(f: A \to B\) então \(g_1 = g_2\). Assim:
    \[g_1 = g_1 1_B = g_1(fg_2) = (g_1f)g_2 = g_2\]
\end{proof}
\begin{prop}
    As seguintes afirmações são verdadeiras:
    \begin{itemize}
        \item Toda identidade \(1_A\) é um isomorfismo e sua própria inversa.
        \item Se f é um isomorfismo, então \(f^{-1}\) também é, tal qual
        \((f^{-1})^{-1} = f\).
        \item Se \(f \in \Hom_\cat (A, B), g \in \Hom_\cat (B, C)\) são
        isomorfismos, então sua composta \(gf\) também é, de inversa
        \(f^{-1}g^{-1}\).
    \end{itemize}
\end{prop}
\begin{proof}
    Basta inspecionar as afirmações.
\end{proof}
\begin{example}[Grupóide]
    De maneira lúdica, podemos pensar em um grupóide como uma categoria, tal
    qual todos seus morfismos são iso.
\end{example}

Um automorfismo de um objeto \(A\) de uma categoria \(\cat\) é um iso de \(A \to
A\), tal qual diretamente \(\text{Aut}_\cat (A) \subset \text{End}_\cat(A)\). É
possível verificar de forma direta, que \(\text{Aut}_\cat(A)\) é fechado para
composição, que por sua vez é associativa, além disso contém a identidade, e
toda inversa. 

Assim, para toda categoria e seus objetos, esse conjunto é grupo.

Acima, tratamos da importante noção de isomorfismos, conforme já realizado
anteriormente. 

Partimos, agora, para a definição em termos categóricos de estruturas já
trabalhadas anteriormente.
\begin{definition}[Monomorfismo]
    Seja \(\cat\) uma categoria. Um morfismo entre objetos \(A, B\) é um
    monomorfismo se para todo objeto \(Z\) de \(\cat\) e todos morfismos
    \(\alpha', \alpha''\) entre \(Z, A\):
    \[f \circ \alpha' = f \circ \alpha'' \implies \alpha' = \alpha''\]
\end{definition}

\begin{definition}[Epimorfismo]
    Seja \(\cat\) uma categoria. Um morfismo entre objetos \(A, B \) é dito
    epimorfismo se para todo objeto Z e todos morfismos \(\beta', \beta''\)
    entre \(B, Z\):
    \[\beta' \circ f = \beta'' \circ f \implies \beta' = \beta''\]
\end{definition}

\subsection{Propriedades Universais}

Até agora, o nosso estudo em categorias tratou de diversos casos específicos,
entretanto, como categorias são objetos matemáticos extremamente gerais,
naturalmente a idea de um tratamento mais universal se torna presente. Assim,
iniciando com definições amplas:
\begin{definition}[Objeto Inicial]
    Seja \(\cat\) uma categoria. Dizemos que um objeto \(I\) de \(\cat\) é
    inicial em \(\cat\) se para todo objeto \(A\) de \(C\) existe um único
    morfismo \(I \to A\). i.e:
    \[\forall A \in \Obj(\cat) : |\Hom_\cat(I, A)| = 1\]
\end{definition}

\begin{definition}
    Dizemos que um objeto \(F\) de \(\cat\) é final se: 
    \[\forall A \in \Obj (\cat): |\Hom_\cat(A, F)| = 1\]
\end{definition}

Se um objeto satisfaz qualquer uma dessas duas propriedades, dizemos que ele é
um objeto terminal. 

Essa definição extremamente simples, nós traz uma propriedade relativamente
forte. Em linhas gerais, queremos mostrar que são únicos a menos de isomorfismo
único, isto é, formulamos:
\begin{prop}
    Seja \(\cat\) uma categoria, então:
    \begin{itemize}
        \item Se \(I_1, I_2\) são objetos iniciais em \(\cat\), então \(I_1
        \eqsim I_2\).
        \item Se \(F_1, F_2\) são objetos finais em \(\cat\) então \(F_1 \eqsim
        F_2\)
    \end{itemize}
\end{prop}
\begin{proof}
    Sejam \(I_1, I_2\) dois objetos iniciais de \(\cat\) então, \(\exists! f:
    I_1 \to I_2\), \(\exists! g: I_2 \to I_1\), além disso o único endomorfismo
    de \(I_1, I_2\) é a identidade, por definição. Dessa forma \(fg = 1_{I_2}\)
    e \(gf = 1_{I_1}\), assim \(f: I_1 \to I_2\) é um isomorfismo.

    De mesma maneira, podemos concluir o mesmo fato para objetos finais.
\end{proof}
Objetos terminais desempenham um papel fundamental em certas propriedades
universais. Os seguintes tópicos exibirão isso (em geral ligamos propriedades
universais com objetos terminais).
\subsubsection*{Quociente}
    Seja \(\sim\) uma relação de equivalência em um conjunto \(A\). Queremos
    entender a afirmação "O quociente \(A/\sim\) é universal em relação a
    propriedade de mapear \(A\) para um conjunto de forma que elementos
    equivalente possuem a mesma imagem, isto é:
    \begin{center}
        \begin{tikzcd}
            A \arrow{r}{\varphi} & Z 
        \end{tikzcd}
    \end{center}
    
    Para qualquer \(Z\) no qual:
    \[a' \sim a'' \implies \varphi(a') = \varphi(a'')\] Definimos assim, a
    categoria \(\cat'\) denotada por \((\varphi, Z)\), queremos saber se essa
    categoria possuí objetos iniciais.

    \begin{prop}
        Seja \(\pi\) a projeção canônica, o par \(\pi, A/\sim\) é um objeto
        inicial dessa categoria.
    \end{prop}
    \begin{proof}
        Considere \(\varphi, Z\) definido acima. Queremos que o diagrama:
        \begin{center}
            \begin{tikzcd}
                A/\sim \arrow{rr}{\overline \varphi} & &  Z \\
                & A \arrow{ul}{\pi} \arrow{ur}{\varphi} & 
            \end{tikzcd}
        \end{center}
        Seja único, i.e \(\exists! \overline \varphi\). Como estamos numa
        categoria, o diagrama comuta tal qual:
        \[\overline \varphi([a]_\sim) = \varphi(a)\] Ou seja, \(\overline \phi\)
        é de fato única para todo \(Z\). 

        Assim, por fim, é preciso que \(\overline \varphi\) seja bem definida.
        De fato:
        \[[a_1]_\sim = [a_2]_\sim \implies a_1 \sim a_2 \implies \phi(a_1) =
        \phi(a_2)\]

        Logo, como queríamos mostrar, esse objeto é inicial.
    \end{proof}
    \subsubsection*{Produto}
    Partindo da ideia de produto de conjuntos, queremos obter um conceito mais
    geral que possa ser observado em qualquer categoria. Com isso, temos a
    seguinte ideia: O produto de dois conjuntos \(A, B\) (com suas respectivas
    projeções naturais) é um objeto final na categoria \(\cat_{A, B}\). 
    
    Dizemos ainda que uma categoria \(\cat\) possuí produtos finitos se
    \(\forall A, B \in \cat, \cat_{A, B}\) possuí objetos finais tal qual
    existam dois morfismos \(A \times B \to A, A \times B \to B\)  

    \begin{exercise}
        Mostre que o produto cartesiano entre conjuntos \(A, B\) obedece essa
        propriedade em \(\cat = \mathfrak{Set}\)
    \end{exercise}

\subsubsection*{Coprodutos}
Intuitivamente pensamos pelo prefixo \textit{co-} que estes seriam o "inverso"
do caso anterior, isto é, mais claramente:

\begin{definition}[Coproduto]
    Definimos como coproduto de dois objetos \(A, B\) de uma categoria \(\cat\),
    os objetos iniciais da categoria \(\cat^{A, B}\), tal qual, no diagrama
    abaixo, para o coproduto \(A \amalg B\) é um objeto de \(\cat\) munido de
    \(\exists! \sigma: A \amalg B\)
    \begin{center}
    \begin{tikzcd}
        A \arrow[bend left]{drr}{f_A} \arrow{dr}{i_A} \\
        & A \amalg B \arrow{r}{\sigma} & Z \\ 
        B \arrow{ur}{i_B} \arrow[bend right]{urr}{f_B}
    \end{tikzcd}
\end{center}
\end{definition}
\begin{prop}
    A união disjunta em \(\mathfrak{Set}\) é um coproduto.
\end{prop}
\begin{proof}
    Definimos a união disjunta de conjuntos \(A, B\) como a união de seus pares
    isomorfos disjuntos, \(A', B'\). Assim, tomando \(A' = {0} \times A, B' =
    {1} \times B\), com \(i_A(a) = i(0, a), i_B(b) = (1, b)\). Note que esse é
    um caso específico, mas não precisamos tratar em geral, já que qualquer
    outra união disjunta será isomorfa a essa definição.

    Assim simplemente da definição, 
    \begin{align*} 
      \sigma(c) = \begin{cases}
        f_A(a), \text{se } c &= (0, a) \in \{0\} \times A \\ 
        f_B(b), text{se } c &= (1, b) \in \{1\} \times B 
      \end{cases}  
    \end{align*}
    Tal qual a comutatividade e unicidade são diretas.
\end{proof}
\end{document}